\documentclass[12pt,letterpaper,english,bibliography=totocnumbered]{scrartcl}

\usepackage{indentfirst}
\usepackage{appendix}
%\usepackage{fullpage}
%\usepackage{subfiles}
\usepackage[T1]{fontenc}
\usepackage[latin9]{inputenc}
\usepackage{color}
\usepackage{babel}
\usepackage{verbatim}
\usepackage{graphicx}
\usepackage[unicode=true,pdfusetitle,
 bookmarks=true,bookmarksnumbered=false,bookmarksopen=false,
 breaklinks=true,pdfborder={0 0 0},pdfborderstyle={},backref=false,colorlinks=true]
 {hyperref}
\hypersetup{linkcolor=blue,citecolor=blue,urlcolor=blue}

%\usepackage{pdfpages}
%\usepackage{comment}

\usepackage[backend=biber,maxbibnames=99]{biblatex}
\usepackage{csquotes}
\addbibresource{references.bib}

% Prevent page breaks within paragraphs
% https://tex.stackexchange.com/questions/21983/how-to-avoid-page-breaks-inside-paragraphs
\widowpenalties 1 10000

\begin{document}

\titlehead{Report on Use of FY2018 UOG-CNAS-EO Core Funds}
\title{Development of a Raspberry Pi Workshop for 4H Students}
\author{Aubrey Moore}
\maketitle

\begin{center}
\includegraphics[width=0.7\linewidth]{UOG-4H-RPi.png}
\end{center}


% The following line provides a link to the source code for this document

\begin{center}
\url{https://github.com/aubreymoore/4HRPi/raw/master/report.pdf}
\end{center}

\tableofcontents{}

\newpage
\section{Background}

I used my FY2019 UOG-CNAS-EO core fund allotment to develop and put on a workshop to introduce local youth to the Raspberry Pi computer and Python coding. This initiative was developed in partnership with UOG-4H following discussions with the UOG-EO Associate Director. 

A Raspberry Pi is a \$35 single board computer which was originally designed to introduce computer science and Python coding to children. However, with exceptional standard features such as HDMI output, WiFi, Linux operating system, and a huge selection of free, open-source software (FOSS), these tiny computers have become very popular among scientists and hobbyists.

This workshop was a trial run to test if we could meet the following objectives:
\begin{enumerate}
	\item Assembly of a complete computer system from a parts kit.
	\item Basic use of a Linux operating system.
	\item Basic Python coding.
	\item Use of the Open Science Framework.
	\item Workshop project: Development of motion sensing security camera using hardware and software from above.
	\item Post-workshop project: Time lapse video of seedling growth.
\end{enumerate}

\section{Activities}
	
Planning for the workshop began in December 2018:

\begin{itemize}
	\item An \href{https://osf.io/ux6jn/wiki/home/}{Open Science Project} was initiated to facilitate documentation \cite{moore_open_2018}.
	\item A \href{https://osf.io/ux6jn/wiki/Workshop%20Outline/}{workshop outline} was created.
	\item A \href{https://osf.io/ux6jn/wiki/Parts%20List/}{parts list} was compiled (7 parts per student; \$193.54 per student) and 12 kits were procured (with some difficulty).
	\item Jupyter notebooks on basic Python coding were compiled \cite{moore_github_2019}.
\end{itemize}

The workshop was run over a total of 20 hours: July 23 through July 26, 2019, 900h to 1300h each day. We used the Science Teaching Laboratory (ALS 124) for this activity. The workshop was led by Aubrey Moore with the assistance of Jeremiah Lorenzo. Only 2 students, Greg Calvo and Dalton Story, were recruited by 4H for this workshop. 
		
\section{Outcomes}

Students achieved objectives 1 through 5.

Neither student completed their \href{https://osf.io/ux6jn/wiki/Time%20Lapse%20Video%20Assignment/}{post-workshop project}, although one of the students corresponded with me after workshop. He asked for and was given advice on how to proceed with his project. But no results were submitted.

\section{Acknowledgments}

Thanks to Dr. Sereana Dresbach for encouragement, support and advice in putting this workshop together; to Megumi Hikichi for help with procurement of parts;  and to Cliff Kyota for recruiting students and for assigning Jeremiah Lorenzo to assist with the workshop. Kimberly Mendiola and Kamille Wang also helped with some sessions.

\printbibliography

\end{document}
